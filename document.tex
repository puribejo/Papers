%
%  $Description: Author guidelines and sample document in LaTeX 2.09$ 
%
%  $Author: ienne $
%  $Date: 1995/09/15 15:20:59 $
%  $Revision: 1.4 $
%



\documentclass[times, 10pt,twocolumn]{article} 
\usepackage{latex8}
\usepackage{times}

%\documentstyle[times,art10,twocolumn,latex8]{article}

%------------------------------------------------------------------------- 
% take the % away on next line to produce the final camera-ready version 
% \pagestyle{empty}

%------------------------------------------------------------------------- 
\begin{document}

\title{Modelling Train Communications with OMNeT++}

\author{Juan-Carlos Maureira\\
INRIA - Sophia Antipolis\\
Email: jcmaurei@sophia.inria.fr\\
% For a paper whose authors are all at the same institution, 
% omit the following lines up until the closing ``}''.
% Additional authors and addresses can be added with ``\and'', 
% just like the second author.
\and
Paula Uribe\\
INRIA - Sophia Antipolis\\
Email: puribejo@sophia.inria.fr\\
\and
Takeshi Asahi\\
Center for Mathematical Modelling\\
Universidad de  Chile\\
Email:tasahi@dim.uchile.cl\\
\and
Jorge Amaya\\
Center for Mathematical Modelling\\
Universidad de  Chile\\
Email:jamaya@dim.uchile.cl\\
}

\maketitle
\thispagestyle{empty}

\begin{abstract}
	This paper presents how to use OMNeT++ to develop a network simulator focused
	on railways environments. Common design problems are analyzed, making emphasis
	in radio communication models. Scalability issues are raised when modelling
	large topologies as they are associated to railways communications.
	Conclusions pointed out model reusability must be re-enforced and a
	component-based design must be adopted in order to produce a tool that could
	deliver valuable results. 
\end{abstract}

%------------------------------------------------------------------------- 
\Section{Introduction}

Network modelling is always a complex problem to address. Abstraction is a
common process to reduce system complexity, and so, make able to implement 
the model. Nevertheless, this practice lead to the question about how
much abstraction is required in order to get concluding results? We present in this
paper our experience in developing a network simulator focused on railways
environment, making emphasis in the required abstraction level in order to
produce usable results to design communication systems in real scenarios. In this 
direction, we propose OMNeT++ as a scalable tool to achieve the objective. OMNeT++ is an 
extensible, modular, component-based C++ simulation library and framework. It support 
model partition in order to distribute the simulation allowing scalability. Additionally, 
there exist several network models (wired and wireless) already implemented in
OMNeT++, with a couple years of refinement, that can be reused to compose a 
simulated model depicting a in-motion network -- on-board of a train at high
speed -- exchanging traffic with an infrastructure network.

%------------------------------------------------------------------------- 
\Section{OMNeT++ and Network Models}

OMNeT++\cite{VaHo08} is a component-based discrete event simulator framework.
It describe a model in terms components that generates events in a chronological sequence. 
Additionally, the composition of components is possible, producing new components, and so,
allowing the model abstraction. There are several network models
already implemented in OMNeT++. The most popular is the INET Framework, which
provides wired and wireless phy layers and several network protocols. Wired
layers are normally well implemented. The same can be observed in
network protocols as IP, TCP or UDP. The problem arises in radio models. 
When modeling radio communication, normally several false axioms are assumed \cite{Kotz04}, 
driving to misleading conclusions. In railways environments, radio
links are normally used to establish communication between a fixed infrastructure and 
the train. Some of these axioms may affect more than others in this scenario.
but, certainly, we can discard a few depending the wireless technology used to
build the link.

%-------------------------------------------------------------------------
\Section{Networking in a Railway Scenario}

Within a railway scenario, networking is usually defined as the need to exchange
traffic between an in-motion network (on-board the train) and a fixed
infrastructure. Several ways to address this issue are proposed:
Satellites, UMTS 3G, Wifi, WiMax, or a combination of them. Each solution
presents pros and cons. But all of them are expensive to test in real
scenarios. Thus, simulation arises as a feasible way to evaluate such
scenarios, iff radio models and scalability issues are addressed. 

%------------------------------------------------------------------------- 
\SubSection{Radio Models}

Radio transmission models are usually described in terms of a transmission
model, a propagation model and an interference model. Modeling accurately all of
them is computationally too expensive, but a too abstract model could guide
to misleading conclusions. OMNeT++ provides a basic radio model (capable to
realize collisions with an additive interference model) and a flexible
one\cite{Kopke08} (allowing a fully customizable radio model). Possible errors
induced by these radio models in the final results, when considering a
railway scenario, in our opinion, are negligible in front of potential
result misestimations caused by the Media Access, soft handover
algorithms or even, the infrastructure network topology. Finally, we propose to
focus de simulator design in describing more accurately the Media Access method
and handover algorithms instead of to model the physical layer as good as
ray-tracing methods\cite{AgFo00} do. Additionally, when IEEE 802.11 are used, an
important issue is to consider interferences\cite{MDD09} due background traffic
produced by adjacent IEEE 802.11 traffic\footnote{The 2.4 Ghz band is crowded of wifi
traffic in urban areas}

%-------------------------------------------------------------------------
\Section{Scalability and Large Topologies}

Railways scenarios normally requires a large infrastructure network topology.
Normally, the in-motion network establish communication with a base station
somewhere near the railway\footnote{metaphorically speaking, since Satellites
are not near the railway.} From there, traffic flows through a backbone network
until it reaches its destination. The delays involved in this transit could be
important when Satellites are used, or throughput may be hard to estimate when
3G networks are used. Modelling all these elements in any simulation tool with
a packet resolution surely will generate an unrunnable model. OMNeT++ provides a
model partition schema allowing to ``split'' the model in order to distribute
the simulation among several processors by using MPI\cite{varga2003efficiency}. In
this way, large topologies can be achieved and executed in a human time-scale.

\Section{Mobility}

Train Mobility is one of the most easy issues to solve in OMNeT++. Following up
the mobility models of wireless networks models, we popose a way-point mobility
schema, which uses a XML-based railway description to describe the train
path, considering speed changes and stop times.  
 
\Section{Conclusions}

%------------------------------------------------------------------------- 
%\nocite{ex1,ex2}
\bibliographystyle{document}
\bibliography{document}

\end{document}
